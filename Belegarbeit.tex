\documentclass[11pt]{article}

\usepackage{textcomp}
\usepackage{sectsty}
\usepackage{graphicx}
\usepackage{titlesec}
\usepackage[onehalfspacing]{setspace}
\usepackage{float}
\usepackage[ngerman]{babel}
\usepackage{fancyhdr}
\usepackage[a4paper, left=2.5cm, right=2cm, bottom=2cm, headheight=2cm]{geometry}
\usepackage{lipsum}
\usepackage{tocloft}

\setlength{\cftsecindent}{0em}
\setlength{\cftsubsecindent}{0em}
\setlength{\cftsubsubsecindent}{0em}
\setlength{\cftsecnumwidth}{4em}
\setlength{\cftsubsecnumwidth}{4em}
\setlength{\cftsubsubsecnumwidth}{4em}


\renewcommand{\sectionmark}[1]{\markright{#1}}
\pagestyle{fancy}
\fancyhf{}
\lhead{{\rightmark }} 
\fancyfoot[R]{\thepage}
\rhead{\includegraphics[scale=0.3]{Logo}}

\usepackage{graphicx}
\graphicspath{ {./Img/} }

\setcounter{secnumdepth}{4}


\setlength{\parindent}{0em}





\begin{document}

\begin{titlepage}
    \begin{center}
        \vspace*{1cm}
        
        \Huge
        \textbf{Infrastructure as Code}
        
        \Large
        \vspace{0.5cm}
        Die Rolle von Ansible für die Automatisierung und Dokumentation von 
        IT-Infrastrukturen
             
        \vspace{1.5cm}
 
        \textbf{Paul Herrmann}
 
        \vfill
                 
        \vspace{0.8cm}
      
        \includegraphics[width=1\textwidth]{Logo_Schule}
             
        reddo IT-SERVICE\\
        BSZ ET Dresden\\
        \today
             
    \end{center}
 \end{titlepage}

\thispagestyle{empty}
\pagebreak
\tableofcontents
\thispagestyle{empty}
\pagebreak
\setcounter{page}{1}
\section{Test}
\subsection{Test Unterkapitel}
\lipsum
\subsection{Unterkapitel Test}
\lipsum
\subsubsection{Test}
\lipsum
\subsubsection{Test}
\lipsum
\section{Test}
\subsection{Test Unterkapitel}
\lipsum
\subsection{Unterkapitel Test}
\lipsum
\subsubsection{Test}
\lipsum
\subsubsection{Test}
\lipsum
\end{document}